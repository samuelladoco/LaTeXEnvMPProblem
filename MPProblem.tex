% Dummy Japanese character `あ' so as to make this file encoded/decoded in UTF-8
\documentclass[12pt, a4paper, papersize, dvipdfmx]{jsarticle}

% ---- Preamble -------------------------------------------------------------- %
% -------- Packages ------------------------------------------------------ %
% AMS
\usepackage{amsmath}

% Enum Item
\usepackage{enumitem}

% Font
\usepackage[T1]{fontenc}
\usepackage{lmodern}

% Font Scale
\usepackage{exscale}

% Link
\usepackage[
    bookmarksnumbered = true, 
    colorlinks = true, 
    setpagesize = false, 
    pdftitle = {{Mathematical Programming Problem (mpproblem) environment}}, 
    pdfauthor = {{@samuelladoco}}, 
    pdfpagelayout = SinglePage, 
]{hyperref}
\usepackage{pxjahyper}

% OTF
\usepackage{otf}


% -------- Environments -------------------------------------------------- %
% New Environments
\newcounter{mpproblem}[section]
\renewcommand{\thempproblem}{\thesection.\arabic{mpproblem}}
\makeatletter
\newenvironment{mpproblem}[1]%
{%
    \protected@edef\@currentlabelname{#1}%
    \par\vspace{\baselineskip}\noindent%
    \ifx#1\empty %
    \else \refstepcounter{mpproblem}$($#1$)$ %
    \fi%
    \hfill%
    $\left|%
    \hfill%
    \hspace{0.00\textwidth}%
    \@fleqntrue\@mathmargin\parindent%
    \begin{minipage}{0.86\textwidth}%
    \vspace{-1.0\baselineskip}%
}%
{%
    \end{minipage}%
    \@fleqnfalse%
    \right.$%
    \par\vspace{\baselineskip}\noindent%
    \ignorespacesafterend%
}%
\makeatother
\newcommand{\mpprobref}[1]{$($\nameref{#1}$)$}

\newenvironment{mpproblem*}%
{%
    \begin{mpproblem}{}%
}%
{%
    \end{mpproblem}%
    \ignorespacesafterend%
}



% ---- Text ------------------------------------------------------------------ %
\begin{document}

\phantomsection
\addcontentsline{toc}{section}{Title}
\title{Mathematical Programming Problem~(mpproblem) 環境}
\author{@samuelladoco}
\date{\today}

\maketitle


\section{概要}
数理計画(数理最適化)問題を、以下のように記述できます。
%
\begin{mpproblem}{$P$}
\label{mpprob:P}
\begin{alignat}{2}
 &\text{minimize}   & \quad  c^\top x  &        \label{eqn:P_Obj}     \\
 &\text{subject to} & \quad  Ax        &\geq b  \label{eqn:P_Con-Eq}  \\
 &                  & \quad  x         &\geq 0. \label{eqn:P_Con-Non} 
\end{alignat}
\end{mpproblem}
%
対応するソースは次のとおりです。
{\small 
\begin{verbatim}
\begin{mpproblem}{$P$}
\label{mpprob:P}
\begin{alignat}{2}
 &\text{minimize}   & \quad  c^\top x  &        \label{eqn:P_Obj}     \\
 &\text{subject to} & \quad  Ax        &\geq b  \label{eqn:P_Con-Eq}  \\
 &                  & \quad  x         &\geq 0. \label{eqn:P_Con-Non} 
\end{alignat}
\end{mpproblem}
\end{verbatim}
}
%
以下のような特徴があります(詳しくは次節以降で解説)。
\begin{itemize}
\item 問題の中身は、amsmathパッケージのalignat、alignを使用可能
    \begin{itemize}
    \item 複数の式の記述が可能
    \item 数式番号の個別付与、相互参照が可能
          (例えば、式\eqref{eqn:P_Con-Non}は非負制約)
    \end{itemize}
\item nameref(hyperref)パッケージを使えば、問題名\mpprobref{mpprob:P}は相互参照可能
\end{itemize}


\section{問題名つきバージョン}
\verb|\begin{mpproblem}{$D$} ... \end{mpproblem}| 
と書くと、 \verb|$D$| の両側に$()$がついたものが問題名として出力されます。
%
\begin{mpproblem}{$D$}
\label{mpprob:D}
\begin{alignat}{2}
 &\text{maximize}   & \quad b^\top y  &        \label{eqn:D_Obj}     \\
 &\text{subject to} & \quad A^\top y  &\leq c  \label{eqn:D_Con-Eq}  \\
 &                  & \quad y         &\geq 0. \label{eqn:D_Con-Non} 
\end{alignat}
\end{mpproblem}


\section{問題名なしバージョン}
\verb|\begin{mpproblem*} ... \end{mpproblem*}| 
と書くと、問題名が出力されません。
\verb||
\begin{mpproblem*}
\begin{alignat}{2}
 &\text{minimize}   & \quad f(x)  &        \label{eqn:FG_Obj}    \\
 &\text{subject to} & \quad g(x)  &\leq 0. \label{eqn:FG_Con-Eq} 
\end{alignat}
\end{mpproblem*}


\section{問題の中身の数式環境}
これまでの例はalignatを使いましたが、もちろん、alignも使えます。
数式が1行の場合は、alignで \verb|&| を使わないで \verb|\quad| などで
スペースを調整するとよいでしょう。
%
\begin{mpproblem}{$Q$}
\label{mpprob:Q}
\begin{align}
\text{minimize} \quad f_1(x_1) + f_2(x_2). \label{eqn:Q_Obj} 
\end{align}
\end{mpproblem}
%


\section{相互参照}
nameref(hyperref)パッケージを使い、
\verb|\begin{mpproblem}{$P$}| の下に \verb|\label{mpprob:P}| とラベルを定義し、
別の場所で \verb|\mpprobref{mpprob:P}| と書くと、
\mpprobref{mpprob:P}と出力されます。
問題名である \verb|$P$| の両側に()がついたものを相互参照できるということです。
さらに、hyperrefパッケージを使うと、
問題名の参照から問題名の定義箇所にジャンプできます。
この\mpprobref{mpprob:P}の$()$の内側の文字をクリックしてみてください
(PDFを拡大したあとでクリックすると、よりわかりやすい)。


\section{課題}
\begin{itemize}
\item 長い問題名を設定すると、縦棒の位置が右にずれてしまいます。
    \begin{itemize}
    \item 縦棒の位置と中身の数式の領域ですが、プリアンブルで \\
          \verb|\begin{minipage}{0.86\textwidth}| 
          と決め打ちしています。 
          数字の \verb|0.86| を調整してください。
    \end{itemize}
\item 問題名なしバージョンでラベルを設定しても、
      コンパイルがとおってしいます。
\item 問題の中身の数式環境は、alignatやalign(同系統のほかものも含む?)
      以外だと、問題の前後のスペースが正しい量にならないようです。
\end{itemize}

\end{document}
